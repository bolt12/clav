% This is samplepaper.tex, a sample chapter demonstrating the
% LLNCS macro package for Springer Computer Science proceedings;
% Version 2.20 of 2017/10/04
%
\documentclass[runningheads,a4paper]{llncs}
%
\usepackage{graphicx}
% Used for displaying a sample figure. If possible, figure files should
% be included in EPS format.
%
% If you use the hyperref package, please uncomment the following line
% to display URLs in blue roman font according to Springer's eBook style:
% \renewcommand\UrlFont{\color{blue}\rmfamily}

%% Useful packages
\usepackage[font=small,labelfont=bf]{caption} % Required for specifying captions to tables and figures

\graphicspath{ {./Images/} }
\usepackage[colorlinks=False]{hyperref} % add links inside PDF files
\usepackage{amsmath}  % Math fonts
\usepackage{amsfonts} %
\usepackage{amssymb}  %
\usepackage{multirow}
\usepackage{float}

\usepackage[portuges]{babel}
\usepackage[utf8x]{inputenc}

\begin{document}
%
\title{CLAV - ESPECIFICAÇÃO E VERIFICAÇÃO DO MODELO FORMAL}
%
%\titlerunning{Abbreviated paper title}
% If the paper title is too long for the running head, you can set
% an abbreviated paper title here
%
\author{Armando Santos \and
Gonçalo Duarte}

%
\authorrunning{A. et al.}
% First names are abbreviated in the running head.
% If there are more than two authors, 'et al.' is used.
%
\institute{University of Minho, Braga, Portugal}
%
\maketitle              % typeset the header of the contribution
%
\begin{abstract}

CLAV é uma plataforma que está a ser desenvolvida pelo Departamento de Informática da Universidade do Minho em parceria com a Direção Geral do Livro, Arquivos e Bibliotecas (DGLAB), e tem como objetivo a classificação e avaliação de todos os documentos que circulam pelas instituições públicas portuguesas. Neste momento existe um modelo do problema especificado em OWL (\textit{Ontology Web Language}), mas tem sofrido várias alterações no decorrer do último ano e não existiu tempo para estudar o impacto dessas mesmas alterações nas pré-condições e invariantes do modelo. Neste projeto, inserido na Unidade Curricular de Laboratórios em Engenharia Informática (LEI) do MIEI/UM, pretende-se formalizar o modelo de raíz assim como os seus invariantes e garantir a consistência dos mesmos, sendo capaz de detetar erros que, até agora, não tenham sido identificados.

\keywords{Administração Pública \and Métodos Formais \and Engenharia Informática \and Alloy Analyzer \and OWL.}
\end{abstract}
%
%
%
\section{Introdução}

Até agora, em Portugal, não existia nenhum sistema de informação que gerisse a classificação e a avaliação dos documentos gerados no âmbito dos processos que circulam dentro das instituições públicas portuguesas. O CLAV veio mudar isso com a elaboração de um catálogo, que se pretende que venha a ser a referência nacional de todos processos da Administração Pública (AP), tendo sido modelado numa ontologia. Esta ontologia está especificada num modelo formal que representa o conjunto de conceitos referentes aos processos de negócio e aos relacionamentos entre eles. Infelizmente, todos os dados e documentação de apoio estão espalhados, desorganizados e em diferentes formatos, o que os torna extremamente difíceis de manter num domínio tão complexo como o da AP. No entanto, embora já tenham sido feitos esforços para criar um formato neutro, como a Macro-estrutura Funcional (MEF), e vários sistemas de exploração e exportação, devido aos problemas associados com a inserção manual dos dados oriundos das diversas instituições e à complexidade e instabilidade dos invariantes e restrições associadas ao modelo não é possível garantir a coerência das relações entre os processos de negócio. Esta incoerência é extremamente crítica uma vez que a classificação e avaliação dos processos possuí legislações associadas e lida com a remoção ou conservação (digital e física) de documentos governamentais.

Deste modo, no contexto da unidade curricular de Laboratórios em Engenharia Informática do Mestrado Integrado em Engenharia Informática da Universidade do Minho e associado ao perfil de Métodos Formais em Engenharia Informática, apresentamos, neste artigo, a especificação e verificação, de raíz, da ontologia e o estudo da coerência dos invariantes e pré-condições que fazem parte dela. Devido à natureza puramente relacional inerente no domínio do problema em questão, iremos tirar partido de métodos algébricos e relacionais para nos ajudar a raciocinar sobre o problema em mãos e utilizar o Alloy \textit{model checker} para nos auxiliar a encontrar falhas no desenho do modelo.

\section{O Problema}

\begin{figure}[H]
\centering
\includegraphics[width=\linewidth]{metamodel.pdf}
\caption{Meta-modelo simplificado}
\label{metamodel}
\end{figure}

Cada instituição pública portuguesa desempenha uma função específica dentro da AP (p. ex. a prestação de cuidados de saúde), associado a cada função existe um conjunto de várias sub-funções (p. ex. a gestão de utentes e serviços clínicos) e cada sub-função possui uma lista de processos de negócio que, concretamente, se materializam em documentos (p. ex. um processo de negócio pertencente à sub-função de gestão de utentes seria o registo clínico de utentes). A Lista Consolidada possui esta estrutura hierárquica de 4 níveis onde os processos de negócio são passíveis de ser desdobrados para efeitos de avaliação como falaremos mais à frente. Cada classe da LC possui um conjunto de atributos que a descreve e a partir do 3º nível (PNs) começam a surgir relações mais complicadas entre processos no campo chamado contexto de avaliação. Este contexto de avaliação, como também iremos ver mais à frente, tem associado um conjunto de invariantes que influenciarão o campo das decisões de avaliação. Este último campo é responsável por conter a informação sobre o Prazo de Conservação Administrativa (PCA) e Destino Final (DF) de um processo que correspondem, respetivamente, ao prazo que o documento deve ser guardado e qual o seu destino uma vez que este prazo expire.

Embora as entidades principais no domínio do problema sejam as classes da LC, existem várias outras que se relacionam direta ou indiretamente com cada uma das classes e que fornecem uma maior profundidade e complexidade ao modelo como podemos observar na figura \ref{metamodel}.

Observando o meta-modelo simplificado, onde a azul se encontram os 4 nivéis de classe, verificamos que este possui uma complexidade natural mesmo sem lhe impor alguma restrição. Sendo assim, e dado o contexto sério em que o problema está inserido, torna-se claro que deve ser feita alguma coisa no que diz respeito a dar algumas garantias acerca da coerência e consistência da LC.

\subsection{Primeiro \textit{Checkpoint}}

Sendo o CLAV um projeto que já existe há 1 ano e já se encontra com alguns componentes operacionais, decidiu-se pegar em toda a documentação existente sobre o modelo e os seus requisitos e, com a ajuda do Professor José Carlos Ramalho, fazer um apanhado de todas as entidades e relações existentes. Durante esta primeira fase, bastantes das reuniões semanais serviram para apurar pequenos detalhes e dúvidas que iam surgindo.

Uma das razões que motivaram este investimento inicial, apesar de já existir uma ontologia definida pela qual nos podíamos guiar, foi a de existir muita documentação solta e incompleta que nem sempre estava de acordo com a versão mais recente da ontologia. Foi então, elaborado um documento atualizado que documenta os mais recentes requisitos e invariantes e que já se tornou bastante útil no refinamento da ontologia original, nomeadamente na eliminação de entidades e relações obsoletas e no apuramento do domínio e contradomínio de várias relações.

Na Secção \ref{SecModel} iremos falar mais detalhadamente sobre cada entidade e relação do modelo e na Secção \ref{SecAlloy} será abordada a respetiva implementação em Alloy.

\subsection{Segundo \textit{Checkpoint}}

Uma das principais motivações deste projeto é estudar a forma como os mais variados invariantes interagem entre si e se de alguma forma se contradizem. É neste sentido que o Alloy, sendo uma linguagem de modelação de software leve que nos permite especificar o tanto o modelo como as restrições a ele associadas, ajuda a detetar erros ingénuos e súbtis.

Após o investimento inicial em colecionar todo o material relevante e necessário sobre o problema, deu-se início ao ciclo de vida de verificação do problema \cite{ref_article1}. Apesar dos invariantes serem abordados com mais detalhes na secção \ref{SubSecInv} é possível adiantar já que foram identificados 38 restrições no total, sendo que 23 dessas foram acrescentadas no tal processo de verificação.

Podemos concluir que o Alloy teve um impacto positivo na análise das restrições do modelo e na especificação do modelo formal. É importante realçar que a utilização de um \textit{model checker} não descarta a necessidade de prova mas é muito útil para encontrar falhas de \textit{design} como podemos constatar. A ausência de contra-exemplos dá uma grande confiança de que uma prova de correção está ao alcance.

\subsection{Avaliação Final}

\section{Modelo Formal} \label{SecModel}

\subsection{Domínio}

%Introduzir cada uma das signatures e o seu contexto no problema.

\subsection{Relações envolvidas}

%Falar nas relações em geral e em especifico as mais importantes envolvidas na classificação e avaliação de PNs

\subsection{Invariantes} \label{SubSecInv}

%Formalizar o modelo e os seus invariantes

\section{Modelação em ALLOY} \label{SecAlloy}

\subsection{Especificação}

%Fazer a ponte entre o modelo formal e o modelo Alloy

\subsection{Verificação}

%Pendente.

\section{Dificuldades}

%Documentação do projeto do CLAV:
%    - espalhada
%    - desorganizada
%    - desatualizada
%    Implicações:
%        - Tivemos que fazer o apanhado de tudo

%Alguns dos invariantes não eram estáveis e sofreram alterações ao longo do projeto assim como alguns pormenores relacionados com o modelo em si.

\section{Conclusão e Trabalho Futuro} \label{SecConclusion}

%
% ---- Bibliography ----
%
% BibTeX users should specify bibliography style 'splncs04'.
% References will then be sorted and formatted in the correct style.
%
% \bibliographystyle{splncs04}
% \bibliography{mybibliography}
%
\begin{thebibliography}{8}
\bibitem{ref_article1}
José N. Oliveira and Miguel A. Ferreira, Member, IEEE: Alloy Meets the Algebra of Programming: A Case Study. IEEE TRANSACTIONS ON SOFTWARE ENGINEERING, VOL. 39, NO. 3, MARCH 2013

%\bibitem{ref_lncs1}
%Author, F., Author, S.: Title of a proceedings paper. In: Editor,
%F., Editor, S. (eds.) CONFERENCE 2016, LNCS, vol. 9999, pp. 1--13.
%Springer, Heidelberg (2016). \doi{10.10007/1234567890}

\end{thebibliography}
\end{document}
